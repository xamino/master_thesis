\chapter{Glossary}
\label{chap:glossary}

\begin{description}[leftmargin=2em,style=nextline,noitemsep,nolistsep]
    \item[Encrypted Connection]
        An encrypted connection is utilized between encrypted peers and between an encrypted peer and a trusted peer.
        Messages are partially modified (encrypted and / or anonymized) to preserve the data privacy.
    \item[Encrypted Peer]
        An encrypted peer contains an encrypted copy of the directory.
        The keys for encryption are not shared with it.
        All messages it receives contain anonymized or encrypted information (for example file names) wherever necessary.
    \item[Fourth Party]
        We define fourth parties for the purpose of this work as follows: for a first party communicating with a second party using a service provided by a third party, fourth parties are those that partake in this communication without the first or second party knowing of it.
        Specifically this includes governmental agencies such as the National Security Association or hostile entities such as industrial espionage.
    \item[Nonce]
        A nonce is an arbitrary number that should be used only once.
        Nonces play important roles in cryptographic practices.
        In Tinzenite nonces mostly serve to avoid replay attacks by making each encryption unique even if the same content is reencrypted.
    \item[Third Party]
        A third party is defined as a party that offers a service for the first party and second party to communicate.
        This can range from a messaging service such as Skype to a service provider such as Dropbox.
    \item[Tinzenite]
        The name of the core software library of this thesis.
    \item[Trusted Connection]
        A trusted connection is only between two trusted peers.
        All information that flows over it is complete and in clear text.
    \item[Trusted Peer]
        A trusted peer has access to the encryption keys and contains an unencrypted copy of the directory.
\end{description}
