\chapter{Conclusion}
\label{chap:conclusion}

This chapter serves as the conclusion of our work, both the theoretical part and the practical proof of implementation.

\section{Theoretical Work}
\label{sec:Theoretical Work}

We have developed and discussed the design decisions for a peer to peer encrypted file synchronization software.
Before defining our concept we looked at existing work and academic papers to help us define the features we would like to cover in chapter~\ref{chap:Related and Existing Work}.
In chapter~\ref{chap:concept} we discussed the general scope of the undertaking and defined the concept of this work.
We combined most features users have come to expect from any other available data storage service with features required to retain the security of the users' data into an encompassing concept.
Chapter~\ref{chap:architecture} thus served to define the actual architecture we would implement as a proof of the concept of the basic functionality.
We discussed how we proposed to define the storage space to work for all required use cases and how the peers would communicate between themselves.
We then discussed our implementation of the proof of concept work in chapter~\ref{chap:Implementation}.
We went into detail in how our implementation is structured and how we implemented core aspects of the proposed architecture.
Finally we wrote about the finished implementation in chapter~\ref{chap:Results and Recapitulation}.
We explained the current feature state of the implementation in comparison to existing work.
Furthermore an outlook on the multiple possible future improvements and expansions was given.

\section{Implementation Work}
\label{sec:Implementation Work}

A large part of our work was the implementation of a proof of concept for the proposed Tinzenite software suite.
This includes example user programs for both the trusted peer and the encrypted peer.
All of our code can be found in the Github organization for Tinzenite~\cite{web:site:github:tinzenite}.
% The code is heavily commented and a publicly viewable version of the documentation can be found at the following locations: encrypted, shared, core, server, tin, model, channel, and bootstrap. TODO: link https://godoc.org/github.com/Tinzenite/{...} correctly

Beyond providing a proof of concept implementation which covers the essential scope, we also used tools, a programming language, and  libraries new to us for our work on Tinzenite.
Thus we expanded on their implications in section~\ref{sec:Tools and Environment}.
Our implementation language of choice was Golang.
We built the communication of Tinzenite on the Tox communication protocol, thus allowing Tinzenite to build on an existing end to end encrypted communication standard.
For the encrypted peer we implemented a storage interface that allows server clients to store the user data in any available storage system.
We included two example implementations for this interface: a simple one that writes the data to disk and another one that allows the server client to write user data to the Hadoop distributed file system.

\section{Closing Statement}
\label{sec:Closing Statement}

We have shown our results of developing and implementing a data synchronization library for secure peer to peer data storage.
This includes preparatory work not only by comparing existing services and academic papers but also developing a new protocol for data exchange.
This library, named Tinzenite, was subsequently implemented as a proof of concept and two example client implementations were also developed, a trusted and an encrypted peer.
While various aspects could still be improved and expanded on, the basic scope of the thesis has been completed.

Retrospectively we are satisfied with the promise of Tinzenite.
We consider our work to be usable for academic, exploratory, and developing purposes but would refrain from utilizing it in a real world scenario due to outstanding security and improvement considerations.
