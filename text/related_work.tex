\chapter{Related and Existing Work}
\label{chap:related}

This chapter serves two purposes.
First we will discuss existing solutions and how our proposed implementation differentiates itself from them.
Then we will take a look at the academic side of file synchronization and discuss related papers.

\section{Existing Software}

For any interent technology two options exist how to struture its architecture, broadly speaking.
Most of the internet as it is today is cleanly divided in a client server structure where a client always request information from a central server.
This is a strongly hierachical structure.
The opposite of this is a distributed, peer to peer model, where a client requests information from any other client and in turn will also respond to queries from other clients.
Both options have been used in existing file synchronization solutions.
Therefore we have divided the existing software we will shortly discuss in the following sections between these two extremes.

\subsection{Client Server Solutions}

For client server solutions any user must rely on the availability of the central server.
These are often hosted by third parties in a distributed manner for reasons of performance and scaling.

\subsubsection{Dropbox}

One of the most popular cloud storage providers currently, Dropbox~\cite{web:site:dropbox}, is known by many users looking for a solution that works across multiple operating systems.
Positive features include web access to stored data, clients for many different platforms, easy sharing of data with outsiders, and ease of use.
On the negative side the service relies on its backend servers, although computers can synchronize files between themselves on a local area network.
Dropbox also lacks any end to end encryption, but does encrypt data while in transit and when "at rest" in their data centers\footnote{See \url{https://blogs.dropbox.com/dropbox/2014/06/weve-got-your-back/}, retrieved on May 14, 2015}.
Therefore it comes with little surprise that they were prominently featured in the Snowden leaks\footnote{See \url{http://rt.com/news/195244-snowden-rid-dropbox-privacy/}, retrieved on May 14, 2015}.
The company does have full access to any data the user uploads to their servers, as long as they do not encrypt the files beforehand themselves.
This in turn means that they can offer up said data if required to by a government, as is the case in the United States of America.

Dropbox offers a free account for a basic storage plan.
Additional storage is available for purchase.
None of the core applications has been open sourced to date, meaning even if Dropbox implemented strong encryption it could not be independently verified.

\subsubsection{Google Drive}

Google Drive~\cite{web:site:gdrive} is similiar to Dropbox from its functionality.
It does go a step further than Dropbox by integrating tightly with their suite of online applications for creating and editing documents, the so called Google Docs.
As can thus be quickly seen Google also has full access to all data that users upload to their servers.
This in turn means that under the PRISM program by the NSA, all such data is also retrievable by a fourth party\footnote{See \url{http://rt.com/usa/nsa-secretly-access-yahoo-google-982/}, retrieved on May 14, 2015}.

Google also offers up a basic storage amount at no charge for the user.
Again additional space can be purchased on a monthly basis.
Apart from offering access via fewer clients than Dropbox, Google also has failed to open source the components of its service.

\subsubsection{Boxcryptor}

Both already mentioned services are thus not to be trusted with private information.
However solutions for encrypting the data sent over such services exist.
One example of such a service is the Boxcryptor software~\cite{web:site:boxcryptor}.
It encrypts all data before uploading it to the connected cloud storage service and decrypts it when retrieving it.
The user keys are attached to the user's account.
Assymmetric encryption is used to upload all keys to the companies key server so that other clients can retrieve and, with a correct password, decrypt the data.
Sharing of data is still possible even for users without an account by utilizing special keys.

The downside to this approach is mainly that it has to be used on top of an existing cloud storage service, althouh it works on most existing services.
However the software is again not open source and thus not independently verifiable.
Boxcryptor also offers a free version of its service, but to access all security features a paid subscription is required.

\subsection{Peer to Peer Solutions}

TODO: Explain how peer to peer solutions work.
Notice that DHT should be briefly highlighted here already.

\subsubsection{BitTorrent Sync}

Negative: Closed source, peers can't be encrypted copies, ...
Positive: no central point of failure, no strict internet requirement, ...
Cite~\cite{web:site:bittorrent_sync}.

\subsubsection{Syncthing}

Negative: peers can't be encrypted copies, ...
Positive: no central point of failure, no strict internet requirement, protocol is mixed data transfer and communication in one with security stuff added, ...
Cite~\cite{web:site:synthing}.

\section{Papers}

%TODO look for papers that relate to my work

Basic file sync stuff \cite{balasubramaniam1998file}.
We: ignore links, file permissions, and unlike the paper file type (we never modify the file itself)!
Detect updates definitions – look at them and point to how we want to do it.
Reconciliation: we'll keep conflicts and propagate as new files, leaving the user to manually sort it out.
Ideally, we'll offer assisstance however (mark them in the dir?).
Insert/delete ambiguity!
I might be able to synthesize rules from this, for example: if a deletion conflicts, leave it be but mark (and apply updates within it if appropriate).
RUMOR --> sounds almost like mine, look at it!
I also use pull method for updates (look at what the gossiping stuff is about maybe?).
What happens for my spec when a change results in the same content?
Theoretically nothing because version+1 and content hash is the same. :D

Paper on algebraic file synchroniser \cite{ramsey2001algebraic}.
Might be interesting: "We identify useful alternatives for conflict resolution, including alternatives that enable users to recover from conflicts by making changes at a single replica."
The algebra of commands is interesting for me (possibly without derive because files are atomic for us?).
Note the following text where they discuss the implementation view of it!
RENAME replaced with MOVE, DERIVE can't be differentiated from EDIT, MOVE replaced with REMOVE and CREATE.
Having a lot of creates is good for my system because it resets version numbers!

This paper explains why we want p2p sync \cite{reiher1996peer}.
"The cost of using peer-to-peer replication, rather than client/server replication, is complexity of the algorithms used to control the replication."

This paper discusses the insert delete ambiguity \cite{page1998perspectives}.
See page 6 for it.
"This resolves the insert/delete ambiguity at the cost of creating a garbage-collection problem: when is it safe to discard a logically deleted directory entry?"
And here is where I got stuck: "Successfully eliminating insert/delete ambiguities requires that prior to discarding a logically deleted directory entry, a directory replica must not only know that all replicas are marked deleted, but further must know that all other replicas are also aware of this fact."
And the following paragraph explains the soluton: "
The garbage-collection algorithm proceeds in two phases.
Phase one compiles the list of replicas that know the entry is deleted.
Phase one ends and phase two begins at a replica when its list is complete, i.e., includes all replicas of the entry.
Phase two compiles the list of replicas that are known to have finished phase one, and concludes when this second list contains all replicas.
When phase two completes at a node, that node knows that all replicas know that all replicas have marked the entry deleted, and therefore it is safe to garbage collect the deleted entry.
Any other node that ever asks about the status of that entry will get the response "entry unknown" from which it can correctly conclude that garbage collection has finished at that site, and hence can finish at the inquirer as well.
There is no ambiguity since the inquirer knows, by virtue of being in phase two, that the other site once knew about the entry and its deletion.
Further discussion of these algorithms is available elsewhere.
"
TODO: look at that solution and see if I can use it (what happens when new peers are added while the algorithm is running? Can I do this for all deleted objects at once (ie for the list) insteand of per object?).
ALSO: page 7 has thoughts on remove / update conflicts.
Basically they remove the dir and move the orphaned file to a special dir\footnote{TODO: make this a hidden dir and also use for conflicts? The dir is handled normally by the sync apart from how files land in it. Clients can then work on conflicts everywhere and the solutions will be propagated correctly.}.
Also note the no updates lost policy!
FINALLY: also note the name conflicts part!

\section{Conclusion}

TODO: list what we synthesized from all the related work here.

TODO: synthesize the feature I want from Tinzenite by going through a comparison of the already existing services (Bitsync, Syncthing, GDrive, Dropbox, etc).
Highlight what they do right and what they do wrong.
I'll expand on these features in the implementation chapter.

TODO: maybe also look at \url{http://www.symform.com/}.
