\chapter{Related and Existing Work}
\label{chap:related}

%TODO write intro

\section{Existing Software}

TODO: synthesize the feature I want from Tinzenite by going through a comparison of the already existing services (Bitsync, Syncthing, GDrive, Dropbox, etc).
Highlight what they do right and what they do wrong.
I'll expand on these features in the implementation chapter.

\subsection{Server Client Solutions}

TODO: Explain how server client solutions work.

\subsubsection{Google Drive}

Negative: Central server, no encryption, ...
Positive: web access, easy to use, integraton with Google Docs...

\subsubsection{Dropbox}

Negative: Central server, no encryption, ...
Positive: web access, easy to use, ...

\subsubsection{Boxcryptor}

TODO: Especially interesting also because of the key distribution model – I might borrow some of those ideas.
Negative: must be used on TOP of large data storage service
Positive: encrypted, sharing possible, etc.

\subsection{Peer to Peer Solutions}

TODO: Explain how peer to peer solutions work.
Notice that DHT should be briefly highlighted here already.

\subsubsection{BitTorrent Sync}

Negative: Closed source, peers can't be encrypted copies, ...
Positive: no central point of failure, no strict internet requirement, ...

\subsubsection{Syncthing}

Negative: peers can't be encrypted copies, ...
Positive: no central point of failure, no strict internet requirement, protocol is mixed data transfer and communication in one with security stuff added, ...

\section{Papers}

%TODO look for papers that relate to my work

Basic file sync stuff \cite{balasubramaniam1998file}.
Ignore links!
Detect updates definitions – look at them and point to how we want to do it.
Reconciliation: we'll keep conflicts and propagate as new files, leaving the user to manually sort it out.
Ideally, we'll offer assisstance however (mark them in the dir?).
Insert/delete ambiguity!
I might be able to synthesize rules from this, for example: if a deletion conflicts, leave it be but mark (and apply updates within it if appropriate).
RUMOR --> sounds almost like mine, look at it!
What happens for my spec when a change results in the same content?
Theoretically nothing because version+1 and content hash is the same. :D

Paper on algebraic file synchroniser \cite{ramsey2001algebraic}.
