\chapter{Concept}
\label{chap:concept}

The following chapter discusses all conceptual work that went into creating Tinzenite.
We will first give an overview of the basic goals of the work created by this thesis.
Next we will expand on the goals by discussing the features we would like Tinzenite to have and defining their scope.
These will be based in part on the existing solutions discussed in the related work.
Based on the concept and the proposed features we will explain the software components we plan to implement.

\section{Basic Goals}
\label{sec:Basic Goals}

The stated goal of Tinzenite is to offer a peer to peer solution for file synchronization that builds on the strengths of Tox.
It is important to us to build the system in a way that it is secure from unauthorized access by third parties, even if they retain a copy of the data.
In fact we propose an explicit client for third party support so that third parties can offer a storage peer as a service.

Therefore we will need to develop a protocol for a decentralized and distributed system that relies on the underlying secure channel.
Based on this we hope to implement a proof of concept peer client for normal computers and a third party client that securely stores the user's data off site.

As proof of the protocol we will implement both programs in Golang.
These should cover the basic requirements for file synchronization and work correctly for the base cases.
Thus we will have an actual implementation against which we can compare our proposal against existing solutions and highlight derived problems, solutions, and differences.

\section{Features}
\label{sec:Features}

This section will define the features we would like Tinzenite to have, including features that go beyond the scope of this thesis.
Therefore we will classify the features by scope.
The exact features we would like to have have been synthesized from the existing and related work from the previous chapter.

\begin{description}[leftmargin=2em,style=nextline,noitemsep,nolistsep]
\item[File Synchronization Protocol]
    Design of an extensible protocol on which all communication between peers will be based.
    The open specification will allow the development of compatible peers by others, important for the extensibility of the system.
\item[Peer to Peer Architecture]
    The complete software suite should run in a direct peer to peer mode to remove the requirement of third parties to facilitate data exchange and to remove the associated security risk.
    We also hope that this feature will allow clients to synchronize independently of the internet: peers should be capable of utilizing local connections directly.
\item[Secure Transport]
    All communication between all clients should always be fully end to end encrypted.
\item[Third Party Client]
    A dedicated client for untrusted third party servers that holds only an encrypted copy of the data.
\item[Shadow Files]
    It should be possible to avoid having to fetch unwanted files for space constrained clients.
    Dubbed shadow objects, this feature could be important for mobile devices as they run on power and bandwidth constraints.
\item[Delta Updates]
    Since it is wasteful to transfer redundant data when only small parts of files are changed, we would like to add the capability to only send the delta difference between two files.
\item[Object Atomicity]
    We will not touch the content of files, instead treating them as singular objects.
    This should help to guarantee that files are never modified by the system beyond the required operations for synchronization.
    In particular this forbids automatically merging changes in files.
    All conflicts must be resolved by the user: we do not intend to guess the correct resolution strategy for any file type.
\item[Passive Peer]
    Since the third party client already stores all data fully encrypted, support for a passive encrypted peer could be easily added.
    This would allow the user to use storage devices as additional peers which can be activated by pointing Tinzenite at them whenever they are connected.
    Much like using mobile active peers as data bridges this feature would allow passive peers to also serve as data bridges while keeping the data fully secure.
\item[Performance]
    The proposed protocol should allow for clients to run as unobtrusively as possible.
    This includes requiring only the bare necessity of performance for all operations and avoiding redundant work.
\end{description}

Please note that many further features are not dependent on the capability of the Tinzenite system but on the implementation of peers.
An example for this is web access to an encrypted storage: this is a feature that explicitly can be implemented in a secure way\footnote{The key for decrypting the data can be unlocked by the user in the web browser by entering the correct pass phrase. Utilizing the shadow file capability the web application would act as a temporary trusted peer until a user is done with their data access.}.

\subsection{Scope}

In this brief section we will go through the exact scope that this work is to fulfill.
We therefore divide the above features into three categories, ranging from required to have the thesis considered successful to those that can be added as extras if time permits or as future work.
Furthermore we we expand on the actual implementation work we intend for each scope definition.

\subsubsection{MUST Have}
\label{subs:MUST Have}

These features are required for the thesis to be considered basically successful.
This means that the basic fundamentals of the proposed complete scope have been met and are in working order.
Specifically this includes a fully working computer client based on a specified API on which all future work can be built on.
This client must offer the basics required to get the system to work in a user friendly manner from setup through daily usage.
Data transfer between multiple trusted clients must work as expected with collision detection and correct version iteration upon updates.

\begin{description}[leftmargin=16em,style=nextline,noitemsep,nolistsep]
\item[File Synchronization Protocol]
    The core protocol must be fully capable of basic file synchronization.
\item[Peer to Peer Architecture]
    The base client and library must be capable of running without a centralized system.
\item[Secure Transport]
    All communication must be fully encrypted.
\item[Object Atomicity]
    Files must not be modified by the system beyond the modifications required for the synchronization.
\end{description}

\subsubsection{SHOULD Have}
\label{subs:SHOULD Have}

Features in this category are features that built on the MUST have features and are thus not strictly required.
In broad terms this includes two important aspects.
First and foremost is the capability of having a client that only retains an encrypted version of the data.
Built on this the second aspect is the server client that only ever retains an encrypted data set for each of multiple users.
The server also adds the capability of handling multiple users' data on a distributed file system capable of handling the large data size that are to be expected for multiple concurrent users.

\begin{description}[leftmargin=10em,style=nextline,noitemsep,nolistsep]
\item[Third Party Client]
    The capability of supporting encrypted clients should be implemented.
\item[Protocol Extension]
    To enable an encrypted third party client, the core protocol will have to be expanded.
\item[Performance]
    Working on an additional peer type should allow us to improve performance of the protocol.
\end{description}

\subsubsection{COULD Have}
\label{subs:COULD Have}

These features are features that will only be implemented if all previous features have been successfully integrated.
They are not required for the thesis to be considered successful but would be nice to have to fully complete the proposed functionality.
Primary aspects that would be added in this phase are additional clients with differing functionality: a mobile client for Android, a web interface for accessing encrypted server clients, and a passive storage client.
These would require additional protocol extensions to support higher performance and better control over the synchronization, such as shadow files and delta updates.

\begin{description}[leftmargin=7.5em,style=nextline,noitemsep,nolistsep]
\item[Shadow Files]
    Peers could be allowed to only fetch files that the user explicitly wishes to have synchronized on a peer basis.
\item[Delta Updates]
    Transfer times of files over limited bandwidths could be optimized by only transferring the differences between them.
\item[Web Interface]
    Support for accessing an encrypted peer via a website.
\item[Mobile Client]
    Implement a client that can run on an Android device.
\item[Passive Peer]
    Built on top of the functionality required for the third party client we could also implement the capability that Tinzenite can use passive storage as passive encrypted peers.
\end{description}

\section{Software Scope}
\label{sec:Software Scope}

This section is dedicated to differentiating the possible client applications we will implement as reference implementations.
Note that the exact feature set is to be determined by the required development time of each feature and thus might lead to some features not being done with the thesis.
Those features can be implemented at a later time if so desired.

\subsection{Tinzenite Core Library}

The Tinzenite core library is the central reference implementation of the protocol.
It builds directly on the Tox core library and wraps the complete communication of Tinzenite.
Programs that implement the provided functionality will attach callbacks and call public methods.

To keep development of clients as easy as possible while at the same time keeping the protocol consistent between them we will separate the core logic for Tinzenite from any user oriented code.
The core library will not handle writing or reading data from the user's disk: these capabilities will be offloaded to the implementation of the client programs.
This will ensure a maximum of adaptability for clients, meaning that they will not be constrained by the cross platform capabilities of the library itself.
The only limiting factor for porting the library to other platforms is the availability of the required Tox core library beneath it.

The core library will also offer helper functions in a sub package to make developing clients as easy as possible.
These will likely include functions for handling delta updates, encryption, and even update detection.
Note that file system watchers will be outside the scope of the core library as every platform may require an adapted file watching program.

\subsection{Client Peer}

The basic client peer will be developed first and serve to validate the protocol.
The primarily targeted platform is Linux, but we hope to offer Microsoft Windows compatible executables too.
As Golang is not operating system specific porting the peer should not be overly difficult.

This software will be the target of the user's primary interaction with the Tinzenite system.
Therefore we plan on including full coverage for required assistance in connecting peers and setting them up.
The client peer as we will implement it will always be a trusted peer and thus store the user's data in clear text on the disk.
That means that we will have to include file watcher functionality to trigger model updates beyond the capability to receive and apply them.
Advanced features that may be implemented include but are not limited to support for shadow files and a low system footprint so that the client software can be run continuously without negatively impacting the operating system performance.

From the user's point of view the client software will provide an interface to connect to and accept new and existing peers.
On the other hand the removal of peers must also be supported from the client software.
It may also be possible to look into representing the state of synchronization within the file manager so that the user has visual feedback on the directory.
Furthermore quick and easy to understand system status information should also be available to quickly see the health of the synchronization network.
We also plan on offering the option on pausing the peer from synchronizing so that the user can retain control over bandwidth usage.

Finally as an added bonus the client peer can detect conflicts and offer help in resolving them.
Ideally the software would at least highlight conflicts for the user so that they can be resolved in a timely fashion even if the directory consists of a large amount of files.

\subsection{Server Peer}

%TODO some of these below are too specific, some won't be implemented.
TODO: intergrate the following here?
Notable features are that a single instance should handle multiple users' accounts.
Since large data amounts are to be expected, the server client will be capable of integrating the Hadoop distributed file system to directly support that.
Support for commercialization of a service based on this client will be provided.
This includes size and amount distinction based on a per user basis.
For trusted peers this feature requires peers to authenticate themselves against each other to differentiate the level of trust.

The server peer will be the second software that we plan on implementing.
Theoretically we can already rely on the stability and working functionality of the underlying core library for the basic features.
New here will be everything pertaining to the encryption of data.

As previously stated the server peer will be responsible for possibly many simultaneous users.
This means we will use the Hadoop file system instead of direct disk storage for storing user data.
The server peer will also offer us the ability to conduct optimizations into the speed and concurrency of the core library.
As advanced features we will implement shadow files for implementing storage constraints and support for the possible web application temporary trusted peer.

Unlike the client peer however we do not require user interfaces (these will be provided by the service provider for the user) and file watchers as updates can not originate from the encrypted data set.
These encrypted peers will also utilize a different subset of the communication protocol as they operate essentially as blind data dumps.

\subsection{Mobile Peer}

If the allotted time allows we intend to complete the Tinzenite software suite by implementing a mobile client for the Android platform~\cite{web:site:android}.
Apart from the specific touchscreen oriented interface this peer would prove that Tinzenite can run on mobile platforms.
Further work may be done here to ensure that Tinzenite runs in a mobile friendly way.
This implies low power consumption and low bandwidth capabilities.
Therefore support for shadow files and deltas are almost strictly required to meet the needs of a mobile peer.
When first starting up the mobile peer for the first time it may even be beneficial to ask the user whether the mobile peer should pull all data or just immediately mark everything as shadow objects, thus requiring the user to mark files manually to be fetched locally but reducing bandwidth requirements enormously.

\subsection{Passive Peer}

While not in itself a program, support for blind passive data dumps as passive peers is another aspect we would love to try out.
This feature would require of Tinzenite to run the base encrypted peer logic on a given directory without the communication aspect via Tox.
These passive peers would allow easy backups of directories on passive storage media that can be easily updated by connecting it to an active peer.
Depending on the remaining time after implementing the active peers we may extend the core library to support this feature as we do not believe it to require overly complicated further development work.
