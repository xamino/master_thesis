\chapter{Concept}
\label{chap:concept}

The following chapter discusses all conceptual work that went into creating Tinzenite.
We will first give an overview of the basic goals of this thesis.
Next we will expand on the goals by discussing the features we would like Tinzenite to have and defining their scope.
Then we will give an overview on the software we would like to build on.
Finally we will highlight the features and differences of each software aspect of the Tinzenite system.

\section{Basic Goals}
\label{sec:Basic Goals}

The stated goal of Tinzenite is to offer a peer to peer solution for file synchronization that builds on the strengths of Tox.
It is important to us to build the system in a way that it is secure from unauthorized access by third parties, even if they retain a copy of the data.
In fact we propose an explicit client for third party support so that third parties can offer a storage peer as a service.

Therefore we will need to develop a protocol for a decentralized and distributed system that relies on the underlying secure channel provided.
It seems clear that to minimize developer work we will implement a reference software library for the core protocol.
Expanding on this we hope to implement a peer client for normal computers and a third party client that securely stores the user's data off site.

\section{Features}
\label{sec:Features}

This section will define the features we would like Tinzenite to have, including features that for now go beyond the scope of the thesis.
Therefore we will classify the features by scope for reference afterwards.
The exact features we would like to have have been synthesized from the existing and related work (see section~\ref{chap:Related and Existing Work}).

\begin{description}[leftmargin=2em,style=nextline,noitemsep,nolistsep]
\item[Protocol]
    Design of an extensible protocol on which all communication between peers will be based.
    We will specify the protocol as JSON encoded messages that will be sent as text messages via the Tox protocol.
    The open specification will allow the development of compatible peers by others, important for the extensibility of the system.
\item[Peer to Peer Architecture]
    The complete software suite should run in a direct peer to peer mode to remove the requirement of third parties to facilitate data exchange and to remove the associated security risk.
    This includes the server client as it should be capable of exchanging data even with other server clients.
    As the underlying Tox provides direct peer to peer connections this should be a given from the beginning.
    We also hope that this feature will allow clients to synchronize independently of the internet: peers should be capable of utilizing local connections directly.
\item[Secure Transport]
    All communication between all clients should always be fully end to end encrypted.
    This feature is also provided by the underlying communications software.
\item[Third Party Client]
    A dedicated client for third party servers that holds only an encrypted copy of the data.
    Notable features are that a single instance should handle multiple users' accounts.
    Since large data amounts are to be expected, the server client will be capable of integrating the Hadoop distributed file system to directly support that.
    Support for commercialization of a service based on this client will be provided.
    This includes size and amount distinction based on a per user basis.
    Notably this feature requires peers to authenticate themselves against each other if they are trusted peers.
\item[Shadow Files]
    It should be possible to avoid having to fetch unwanted files for space constrained clients.
    Dubbed shadow objects, this feature could be important for mobile devices as they run on power and bandwidth constraints.
\item[Delta Updates]
    Since it is wasteful to transfer redundant data when only small parts of files are changed, we would like to add the capability to only send the delta difference between two files.
    Note that this is not required for the core functionality but a nice to have optimization.
\item[Object Atomicity]
    We believe that the proposed system should concentrate on one thing and do it well\footnote{Also known as the Unix philosophy.}.
    Therefore we will not touch the content of files, instead treating them as singular objects.
    This should help to guarantee that files are never modified by the system beyond the required operations for synchronization.
    In particular this forbids automatically merging changes in files.
    All conflicts must be resolved by the user: we do not intend to guess the correct resolution strategy for any file type.
\item[Client Agnostic]
    As stated we propose to implement the core protocol functionality in a central library that can be used by different clients independent of their concrete purpose and implementation.
    Therefore we will implement the library as platform agnostic as possible, relying on clients to implement the required file operations and system calls as required.
\item[Concurrency]
    Building on one of the original goals of Golang, we plan on implementing the core library of Tinzenite in an efficient strategy by fully utilizing the language's built in concurrency.
    This should allow Tinzenite to make full use of multi core systems and avoid locking the library up should a single process lock up.
\item[Usability]
    Last but not least in any way we hope to implement reference software in a way that any user can, after the initial setup, use Tinzenite without complex interactions.
    Once the peers for a directory are setup Tinzenite should run as non-intrusive as possible while retaining the user's confidence in the reliability of the system.
    An example for a usability feature might for example be a way to easily authenticate newly connected peers via QR codes or word based hashes instead of the full Tox ID.
\end{description}

There are also features that would be nice to have but are beyond the goal of this thesis.
These include for example the ability to share single files within a synchronized directory with outside users, as seen in Google Drive or Dropbox for example.
This feature is excluded because it requires a more complicated security and encryption model that we do not feel capable of implementing securely.
Security is hard to do correctly, therefore we will limit ourselves before jeopardizing the security of the whole system.

Another feature that would be sensible to have is data obfuscation for encrypted peers.
This would make it even more difficult for nefarious services to extract any information from a hosted encrypted peer.
This would however require a more advanced and distinguished transmission mode for encrypted peers.
Furthermore data deniability is most definitely beyond the capabilities of the proposed implementation (and since the core goal of Tinzenite is data availabilty, somewhat contrary).
If a user allows full access to the directory to an untrusted party there is little that we can do to mitigate the security breach.

Please note that many further features are not dependent on the capability of the Tinzenite system but on the implementation of peers.
An example for this is web access to an encrypted storage: this is a feature that explicitly can be implemented in a secure way\footnote{The key for decrypting the data can be unlocked by the user in the web browser by entering the correct pass phrase. Utilizing the shadow file capability the web application would act as a temporary trusted peer until the user is done with his data access.}.
However we make no guarantee that we will get around to implementing it.

\subsection{Scope}

In this brief section we will go through the exact scope that this work is to fulfill.
We therefore divide the above features into three categories, ranging from required to have the thesis considered successful to those that can be added as extras.
Furthermore we we expand on the actual implementation work we intend for each scope definition.

\subsubsection{MUST Have}
\label{subs:MUST Have}

These features are required for the thesis to be considered basically successful.
This means that the basic fundamentals of the proposed complete scope have been met and are in working order.
Specifically this includes a fully working computer client based on a specified API on which all future work can be built on.
This client must offer the basics required to get the system to work in a user friendly manner from setup through daily usage.
Data transfer between multiple clients must work as expected with collision detection and correct version iteration upon updates.

\begin{description}[leftmargin=13em,style=nextline,noitemsep,nolistsep]
\item[Protocol]
    The core protocol must be fully capable of the basic file synchronization.
\item[Peer to Peer Architecture]
    The base client and library must be capable of running without a centralized system.
\item[Secure Transport]
    All communication must be fully encrypted.
\item[Object Atomicity]
    Files must not be modified by the system beyond the modifications required for the synchronization.
\end{description}

\subsubsection{SHOULD Have}
\label{subs:SHOULD Have}

Features in this category are features that built on the MUST have features and are thus not strictly required.
In broad terms this includes two important aspects.
First and foremost is the capability of having a client that only retains an encrypted version of the data.
Built on this the second aspect is the server client that only ever retains an encrypted data set.
The server also adds the capability of handling multiple users' data on a distributed file system capable of handling the large data size that are to be expected for multiple concurrent users.

\begin{description}[leftmargin=9em,style=nextline,noitemsep,nolistsep]
\item[Third Party Client]
    The capability of supporting encrypted clients should be implemented.
\item[Client Agnostic]
    This feature should be a given once we have started implementation of the second peer program.
\item[Usability]
    The base usage of Tinzenite should be trivial for most users from a usability perspective.
\end{description}

\subsubsection{COULD Have}
\label{subs:COULD Have}

These features are features that will only be implemented if all previous features have been successfully integrated.
They are not required for the thesis to be considered overly successful but would be nice to have to fully complete the proposed scope.
Primary aspects that would be added in this phase are a mobile client, most likely as an application for Android, and possibly even a web interface for accessing encrypted server clients.
A further smaller aspect would be shadow file capabilities so that data can be selectively synchronized on constrained devices, specifically mobile devices for example.

\begin{description}[leftmargin=7.5em,style=nextline,noitemsep,nolistsep]
\item[Shadow Files]
    Peers could be allowed to only fetch files that the user explicitly wishes to have synchronized on a peer basis.
\item[Delta Updates]
    Transfer times of files over limited bandwidths could be optimized by only transferring the differences between them.
\item[Concurrency]
    The core library could support communicating with multiple peers concurrently, thus working around bandwidth constraints.
\end{description}

\section{Dependencies}
\label{sec:Dependencies}

This section briefly highlights the technologies we chose to support our work.
These range from software solutions to data organization standards.
Here we will concentrate on the broad features we require for Tinzenite and how they may impact our work.
For our experiences and how exactly we used them see chapter~\ref{chap:implementation} where we go into the specifics of the implementation of Tinzenite.

\subsection{Tox}

A core aspect of this thesis is implementing the system using the peer to peer encrypted communication channel provided by the Tox communication suite~\cite{web:site:tox}.
Initially developed as a Skype replacement the underlying transport layer was also intended to be usable for alternative services.
We will make use of this and let Tox handle most of the communication aspects.

Unlike many other communication applications, Tox does not require the creation of an account online.
A user can create as many Tox identities as desired.
Each identity consists of a public and private key where the public key is the identity of the user~\cite{web:site:tox:crypto}.
Tox identities are dynamically mapped to the user's current internet address whenever the users are online via a distributed hash table.
Once two users are online at the same time (off line messaging is still being implemented at this time) and have added each other as friends (which stores the other's Tox identity for future usage) they can establish a communication channel.
The Tox messaging clients use the channel to facilitate text, audio, and video chats, with support for file transfers.
All data is encrypted with perfect forward secrecy and sent directly from one client to the other\footnote{Although TCP relay tunneling is sometimes used to punch through obstructions.}.

Tinzenite will build on the peer to peer, distributed, and encrypted communication channel provided.
In our case each directory on each peer will be its own Tox identity.
Every directory that is synchronized between multiple devices has its own network of friends which consists of the group of authorized peers.
For the setup Tinzenite will require the user to allow the initial connection to any single other peer.
The friend list is then synchronized by Tinzenite between all peers automatically.

Since Tox itself is written in C and we intend to use Golang we require a binding for the Tox core.
We intend to use an existing binding - specifically this one \cite{web:site:tox:golang} - to allow us to concentrate on the core aspects of our thesis.

\subsection{Golang}

Our implementation language of choice is Golang, usually referred to by the shorthand Go~\cite{web:site:golang}.
The language is new to the author of this thesis but will hopefully provide to be an interesting intellectual challenge.
Since the main language at Ulm University is mostly Java for student work, this thesis will hopefully also offer some insight into the differences between the two.

Golang was created at Google in 2007, announced and open sourced in 2009.
The reason for yet another language is given in the language's frequently asked questions page as: "Go is an attempt to combine the ease of programming of an interpreted, dynamically typed language with the efficiency and safety of a statically typed, compiled language. It also aims to be modern, with support for networked and multicore computing. Finally, it is intended to be fast: it should take at most a few seconds to build a large executable on a single computer."~\cite{web:site:golang:faq}.
Since Tinzenite is a networked system and offers many possible applications for using concurrent operations it was deemed a good match for this thesis.

Golang, much like Java, comes with a wide range of available default packages to use directly within a program.
Beyond the basics we will make use of the provided cypher package to implement all encryption related tasks where required.

\subsection{Hadoop}

As the encrypted server peer is intended to run as a service for multiple parallel different Tinzenite peers of multiple users it will require some extra work into how to store the expected large size of the data.
We will therefore look into using Hadoop~\cite{web:site:hadoop} for data storage.
This will allow third parties to run a single instance of the server peer even for many users at once while being able to reliably service all data.

%TODO
TODO: Need to rewrite / add to this once I've actually used Hadoop.
Currently I'm somewhat uninformed on the exact consequences of using this for the server peer.

\subsection{JSON}

Since the underlying Tox channel is initially text based we propose to implement all peer to peer communication as a human readable messaging format.
We will therefore utilize Javascript object notation, short JSON, as a machine readable message format while retaining easy readability for developers.
As an added bonus Golang has high level support for converting language structures into and from JSON.

\subsection{Librsync}

The feature of delta updates is again not a trivial matter and we thus turn to existing solutions to shorten our work load in this regard.
We intend to utilize librsync~\cite{web:site:librsync} for the implementation of the specifics.

The librsync library is similar to the rsync and xdelta programs.
Unlike the two however librsync does not require the file in both before and after state to create a delta.
Requiring both versions of the file would immediately remove any benefit of using file deltas in our case as we would be required to send the file completely anyway.
Librsync avoids this by only sending a signature of the original version of the file which consists of block-wise hashes of the contents.
By comparing the new version block for block it can then determine which blocks have been changed and send only the corresponding data.

The library is again written in C, therefore requiring a binding for us to be able to directly use its functionality.
We found and plan to use the binding offered at \cite{web:site:librsync:golang}.

\section{Software Scope}
\label{sec:Software Scope}

This section is dedicated to differentiating the possible client applications we will implement as reference implementations.
Note that the exact feature set is to be determined by the required development time of each feature and thus might lead to some features not being done with the thesis.
Those features can be implemented at a later time if so desired.

\subsection{Tinzenite Core Library}

The Tinzenite core library is the central reference implementation of the protocol.
It builds directly on the Tox core library and wraps the complete communication of Tinzenite.
Programs that implement the provided functionality will attach callbacks and call public methods.

To keep development of clients as easy as possible while at the same time keeping the protocol consistent between them we will separate the core logic for Tinzenite from any user oriented code.
The core library will not handle writing or reading data from the user's disk: these capabilities will be offloaded to the implementation of the client programs.
This will ensure a maximum of adaptability for clients, meaning that they will not be constrained by the cross platform capabilities of the library itself.
The only limiting factor for porting the library to other platforms is the availability of the required Tox core library beneath it.

The core library will also offer helper functions in a sub package to make developing clients as easy as possible.
These will likely include functions for handling delta updates, encryption, and even update detection.
Note that file system watchers will be outside the scope of the core library as every platform may require an adapted file watching program.

\subsection{Client Peer}

The basic client peer will be developed first and serve to validate the protocol.
The primarily targeted platform is Linux, but we hope to offer Microsoft Windows compatible executables too.
As Golang is not operating system specific porting the peer should not be overly difficult.

This software will be the target of the user's primary interaction with the Tinzenite system.
Therefore we plan on including full coverage for required assistance in connecting peers and setting them up.
The client peer as we will implement it will always be a trusted peer and thus store the user's data in clear text on the disk.
That means that we will have to include file watcher functionality to trigger model updates beyond the capability to receive and apply them.
Advanced features that may be implemented include but are not limited to support for shadow files and a low system footprint so that the client software can be run continuously without negatively impacting the operating system performance.

From the user's point of view the client software will provide an interface to connect to and accept new and existing peers.
On the other hand the removal of peers must also be supported from the client software.
It may also be possible to look into representing the state of synchronization within the file manager so that the user has visual feedback on the directory.
Furthermore quick and easy to understand system status information should also be available to quickly see the health of the synchronization network.

\subsection{Server Peer}

The server peer will be the second software that we plan on implementing.
Theoretically we can already rely on the stability and working functionality of the underlying core library for the basic features.
New here will be everything pertaining to the encryption of data.

As previously stated the server peer will be responsible for possibly many simultaneous users.
This means we will use the Hadoop file system instead of direct disk storage for storing user data.
The server peer will also offer us the ability to conduct optimizations into the speed and concurrency of the core library.
As advanced features we will implement shadow files for implementing storage constraints and support for the possible web application temporary trusted peer.
Unlike the client peer however we do not require user interfaces (these will be provided by the service provider for the user) and file watchers as updates can not originate from the encrypted data set.

\subsection{Mobile Peer}

Finally and if the allotted time allows we intend to complete the Tinzenite software suite by implementing a mobile client for the Android platform~\cite{web:site:android}.
Apart from the specific touchscreen oriented interface this peer would prove that Tinzenite can run on mobile platforms.
Further work may be done here to ensure that Tinzenite runs in a mobile friendly way.
This implies low power consumption and low bandwidth capabilities.
Therefore support for shadow files and deltas are almost strictly required to meet the needs of a mobile peer.

\section{Security Considerations}
\label{sec:Security Considerations}

As stated in the motivation for this thesis (see section~\ref{sec:Motivation}) security and privacy are a primary concerns for this thesis.
As such we will give a brief overview of conceptual thoughts on the security of the proposed work.
Please note however that a full security review is not the goal of this work or within its scope.
Nonetheless we hope to offer something secure enough for practical usage.

First we will concern ourselves with a network of only trusted peers.
In this case Tinzenite should be very secure as long as the user ensures that no man in the middle attack happens on setting up initial connections between new peers.
All further communications are encrypted and authorized by Tox.
This in turn means that Tinzenite remains as secure as the software it is built on.
As with any software all security is for naught should the user make an error or allow the stored data to leak outside of Tinzenite.
This is especially true as Tinzenite does not offer protection from other programs reading out the stored data in a clear text directory as this would make intended changes increasingly difficult.

So what changes when the server peer or other similarly encrypted third party peers are given access to the network?
Not a lot for the communication: Tox still theoretically guarantees a fully encrypted and authorized data channel.
Since the to be stored data is encrypted on trusted peers before being sent, direct access to the user's data should also prove difficult, pending that the encryption algorithm is secure enough.

One possibility that we must protect against is that third party peers are considered as trusted update propagation points.
Without checks this can be used to introduce bogus updates into the system: random data can be used to overwrite valid files and deletion updates initiated.
The first case can be easily detected without further changes to the core protocol: the inclusion of content hashes within the object model removes this threat.
As the hash is also encrypted before file transfer

Some information is leaked however.
While we take care to encrypt object names and obfuscate their paths by replacing all names with the object identification hashes within the object model, we can not easily obfuscate the directory structure, thus theoretically allowing an attacker to gleam information by analyzing it.
This is only made easier by the fact that file sizes are roughly the same whether encrypted or not, with an accuracy down to the block size of the encryption scheme.
The information leak could be mitigated by implementing the core protocol with encrypted oblivious storage~\cite{goldreich1987towards}: however this may well be worthy of its own thesis and thus not further considered here.

Possibly even more problematic is the clear text storage of the peer list even on encrypted peers as it can be used to determine the size of the user's Tinzenite peer network.
However to allow encrypted peers to facilitate file transfer between two mutually exclusively online peers, it must know this information.
This is mitigated by the fact that Tox IDs are hard to guess and not shared beyond the Tinzenite peer network.
They however can be used to monitor the user's devices by utilizing the Tox distributed hash table to get current internet protocol addresses.
