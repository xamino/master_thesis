\chapter{Results and Recapitulation}
\label{chap:Results and Recapitulation}

This chapter will discuss the proof of concept implementation and recapitulate on how the architecture was implemented.
We will then discuss improvements and future work.

\section{Comparison}

In this section we will briefly and informally compare Tinzenite against existing solutions.
Note that an actual study comparing our solution against other storage providers was not the intended goal of this thesis\footnote{TODO TAMINO: bad, reword.}.

\subsection{Performance}
\label{sub:Performance}

Tinzenite shares a large advantage with the other discussed related peer to peer solutions.
Because the peers communicate directly with each other the synchronization speed for peers within the same local area network is substantially higher than the solutions that must first upload data to a remote server.
In the local area network where Tinzenite was mainly developed the only speed limit was that of the wireless internet for data transfers.

When transferring data between two peers over the internet at large Tinzenite is restricted by the broadband upload speed of the user's internet connection.
While generally speaking the upload speed is only a fraction of the download speed for most consumer connections, Tinzenite does not suffer under this.
This is because when comparing transferring a file via Tinzentie and any server hosted service, both require an initial upload of the file.
Tinzenite actually wins in terms of speed because by the time the upload has completed, the other Tinzenite already has the complete file downloaded while the alternative via the server must still download the file.

%TODO: add some sort of example, possibly in graph form?
TODO: add some sort of example, possibly in graph form?
Nah, just add a text example.
For that move a large file between two devices with Tinzenite and with GDrive.

TODO: do an informal comparison to previously highlighted comparable software.
Should include hard facts such as speed (internet vs intranet) and limitations.

\subsection{Usability}
\label{sub:Usability}

Here Tinzenite in its current state definitely loses out to existing solutions.
While actually running the software requires little to no user interaction the setup and management of peers is currently only available via command line interface programs.
While nothing speaks against implementing a graphical user interface in principle, this was not an essential part of our work no such client was implemented due to time constraints.
And while Tinzenite theoretically supports web access to encrypted peers (see section~\ref{sub:Additional Peer Versions} below), this capability was also not tested or implemented.
We believe that web access would improve usability by a large margin and allow Tinzenite to close the feature gap.

From a technical standpoint Tinzenite is feature complete in that it can and does synchronize directories correctly.
However further work should be done for edge cases where the usability could be improved, such as for merge conflicts.
By notifying and offering support in resolving merge conflicts Tinzenite could provide a more active support for its usability.
Such features could easily be integrated into a client program without requiring modifications to the underlying Tinzenite code.

A feature important for the security of a Tinzenite network is support of clients for strong passphrases.
User's should be guided to create long and secure passphrases as they then become increasingly harder to guess.

\subsection{Security}
\label{sub:Security}

Tinzenite's security is built on one of the most scrutinized encryption libraries currently in existence, the NaCl library.
Thus the encryption of data that untrusted, encrypted peers receive should be more than sufficient to deny unauthorized third party access.
Should services running encrypted peers be requested by hostile governments to hand over user data said data would be inaccessible without brute forcing the passphrase used to encrypt the encryption keys.
Thus the security of the user data primarily depends on the user's willingness to use a sufficiently complex passphrase.

Theoretically this feature is Tinzenite's distinguishing point between all existing implementation mentioned in section~\ref{sec:Existing Software}.
Like other peer to peer solutions Tinzenite works directly between peers with most of the advantages that brings with it.
Tinzenite allows off site backup of the data however.
Unlike the server-client solutions Tinzenite does not require an account or trust in a third party.
We believe that Tinzenite therefore combines the best of both worlds in regards to security while retaining readily available access to user stored data for authorized entities.

%disclaimer :P
However Tinzenite should not be trusted with secure data yet.
Primarily because it was written as a proof of concept and thus is probably not bug free nor are we confident enough of our implementation of the encryption scheme.
Furthermore Tinzenite should only be trusted after a security audit has been performed on the core logic.
Tinzenite's open source nature however allows any interested parties to audit the code at their own leisure.
Furthermore Tinzenite does leak some information: namely the peer list is available unencrypted.
While a small risk as no true user information is stored in those files it is still a possible attack vector for a malicious peer.

\section{Future Work}
\label{sec:Future Work}

Tinzenite offers a lot of room for future work, both concerning the current implementation and expanding on the provided work.
This section will serve to discuss many of the points we believe could greatly benefit any further work on Tinzenite.
Thus we will begin this section by discussing changes to the current Tinzenite architecture and implementation.
We will conclude this section by offering an outlook of further work that could be based on the existing implementation.

\subsection{Improvements to Existing Work}
\label{sub:Improvements to Existing Work}

Since our implementation of the architecture was built from the ground up without previous knowledge of many of the aspects touched on by this work, we encountered a number of things that should and could be improved.

\subsubsection{Data Transfer}
\label{subs:Data Transfer}

We believe our implementation to be not fully optimized for the best possible data transfer characteristics between a peer network.
Thus improvements to how Tinzenite fetches and sends data could be implemented to improve robustness and speed of data transfers.

In our implementation when a trusted peer receives an update, the associated file is fetched from the peer where the update originated from.
If the update is received from multiple peers only the first peer is used to fetch the data.
Building on the same ideas that led to the development of the torrent protocol, we could implement swarm fetching capabilities.
This would allow the unchoking of saturated peers where upload speeds are slow and allow the file request to complete even if one peer out of many goes offline or encounters issues.

Issues may arise for implementing this for two reasons.
If delta fetching is implemented as described below, care has to be taken to keep the swarm behavior compatible even if multiple peers have varying states of the original file.
Another source of possible issues is how to combine it with the existing encrypted peer behavior.
Since encrypted peers are currently locked to a single trusted peer for a synchronization this precludes having them partake in a swarm apart from the issue that the encrypted peer will transfer encrypted data while trusted peers will transfer unencrypted data.

\subsubsection{Peer Behavior}
\label{subs:Peer Behavior}

Trusted peers of the current Tinzenite implementation synchronize with timing based intervals.
While this works sufficiently for the proof of concept, ideally it would not be the case.
Instead peers should dynamically adjust the timing and order of operations for when to synchronize based on the network environment and current status of the own directory.

An example for this includes pausing the remote synchronization interval if outside changes are still being fetched to avoid unnecessary double fetch requests and associated work.
Another example would be extending the Tinzenite \emph{core} package to allow for finer control over which peers to synchronize and then implementing the client program so that peer synchronizations happen in a smarter fashion (or implementing this directly within Tinzenite itself).
This could include synchronizing only once when initially connected and then simply updating locally, avoiding unnecessary complete model comparisons.
Synchronization with encrypted peers could also be improved to avoid locking multiple encrypted peers at once and ensure that they are kept up to date at a reasonable rate.

\subsubsection{Delta Updates}
\label{subs:Delta Updates}

Fetching a file in Tinzenite currently always transfers the complete binary data, even if only a small part of the file was changed between two versions.
Thus an improvement would be to implement that Tinzenite only sends the differences between two versions of a file between peers.
We propose to use rsync algorithm for this~\cite{tridgell1996rsync}, specifically the librsync implementation~\cite{web:site:librsync}.
The required information for the library to work can be integrated into the existing request messages.
Delta updates could however only be used between trusted peers since the encrypted data is completely different for every change.
Only needing to send the changed part of a file should increase the speed of Tinzenite transfers immensely, expecially for large files.

\subsubsection{Server Peer}
\label{subs:Server Peer}

The current implementation for the encrypted peer was implemented for a single Tinzenite network instance.
It could be extended to provide service for multiple Tinzenite networks and multiple users.
User accounts can be differentiated by reading parts of the authentication file: the user name can be checked with the bcrypt hash.
Care should be taken to ensure that the user does not provide the same access password than the passphrase used to access the Tinzenite encryption, although this won't be enforceable by Tinzenite.

Another feature that the server client should be capable of is enforcing potential size restrictions.
This feature may also be used for a future mobile client as described in section~\ref{subs:Additional Peer Versions}.
What this means is that Tinzenite should support clients refusing to fetch additional files to enforce a specified size of a directory.
It could be implemented by building on the shadow files capability which we will expand on in the next section.
The interesting case is of course what happens to files that are above the limit after a modification: we propose either making the file a shadow file as soon as it crosses the limit or allowing modifications to push the size above the limit temporarily.
For encrypted peers the enforcement of size restrictions must be handled by the trusted peers.
This in turn means however that encrypted peers must be capable of denying additional updates since trusted peers could not be trusted to correctly enforce a size limitation.

\subsubsection{Shadow Objects}
\label{subs:Shadow Objects}

Depending on the location of a client a user may wish to only access specific files without having to get an entire set or updates.
This is a nice feature to have in the case of space and bandwidth restricted devices such as mobile devices or for the web interface.
This feature could also be used to prioritize which objects Tinzenite will fetch first.

Functionality for the shadow file feature is available via the currently unused \textit{"shadow"} attribute.
It affects only files directly as the creation of directories is not significant from a size point of view.
The attribute only serves as a shortcut to set all files of a directory implicitly to being shadowed.
If files are marked as shadow files they are not updated on the disk, only their model.
By setting the shadow flag to true the client will then immediately try to fetch the binary file from connected and available peers.

Shadow files pose a few additional difficulties that would have to be solved however.
First and most trivial: what happens to an already synchronized file when the shadow attribute is set?
We propose that the file is immediately removed although this could be expensive in terms of bandwidth if the users quickly change their mind again as the file must then be fetched again all anew.
A more sophisticated approach would integrate the size restriction capability of the client as proposed previously.
By setting the space limit to a number below the full size of the directory files will only be immediately removed if near the space limit.
If the user changes their mind the file may thus still be immediately available.

So what do peers do if they receive a model update where the shadow flag is set?
It is important to note here that the shadow flag is considered to be transient when synchronizing models, meaning its value is considered to be local only.
However it is still sent as it is used to determine for the receiving peer whether it can fetch an update from the other peer if applicable.
Again it is up to the peer what happens upon receiving a shadow file update: trivially a peer that has a non shadow copy of the file will ignore shadow updates as it can not fetch the binary file update successfully from it.
It will then have to wait for another peer to offer the update where the attribute is not set.

%TODO: explain that we need to disable shadow files on the desktop peers. Or extend the protocol to avoid losing data.
The final edge case is a challenging one: what Tinzenite does not provide is a way to ensure that one full copy of the shadowed file is always kept somewhere.
If the user marks a file as shadowed on all peers it may well happen that Tinzenite loses the file.
For now we propose to avoid this by explicitly warning the user of this possibility.
One way to mitigate this risk is by allowing user defined shadowed files only for specific clients: we can probably assume that any full desktop peer should always retain a full copy of the directory anyway.
Thus shadow files should only be used for the proposed mobile peer and web interface peer.

\subsubsection{Implementation Improvements}
\label{subs:Implementation Improvements}

Our current implementation, while working, is surely not the best way to implement it.
Not all possible error cases are handled in the most optimal way possible.
Furthermore the implementation of Tinzenite could use extensive testing and debugging.

\subsection{Expanding Work}
\label{sub:Expanding Work}

Tinzenite offers a lot of room to build on.
In this section we will discuss some proposals for building on our work instead of modifying it.
Note that some expanded features require features from the previous section.

\subsubsection{Encrypted Peer Synchronization}
\label{subs:Encrypted Peer Synchronization}

TODO: write a section on allowing encrypted peers to synchronize between themselves.
Would have to happen "blindly" but should work.
Note that this requires a shared state because of possible merge conflicts!

\subsubsection{Data Obfuscation}
\label{subs:Data Obfuscation}

While encrypted peers already only store encrypted files, simply encrypting a file may not be sufficient to prevent meta information collection on the directory contents.
Thus we propose that future work could include modifying the encrypted peer implementation and the transfer protocol so that trusted peers send not only encrypted but also obfuscated data to encrypted peers, effectively implementing an oblivious storage system~\cite{goldreich1996software}.

This especially makes sense if combined with the swarm behavior mentioned previously and the encrypted peer synchronization.
The entire group of encrypted peers could then be used to obfuscate and store data redundantly.
This would increase the third party security and further reduce the trust of said party required.
Obfuscation would be sufficient to hide most meta data that may be deduced from encrypted synchronization.

\subsubsection{Additional Peer Versions}
\label{subs:Additional Peer Versions}

Our current implementation of Tinzenite offers two peers: a standard trusted peer for a personal computer and a peer for encrypted storage.
As discussed in section~\ref{sec:Software Scope} we have already considered adding a mobile client for smartphones and a web interface client for temporary access to an encrypted peer, plus a passive peer for secure cold storage.

A mobile peer would be a trusted active peer of the Tinzenite network.
As previously discussed however the mobile peer would most likely not retain a full copy of the Tinzenite directory due to size and bandwidth limitations.
Thus both shadow objects and size restrictions are likely prerequisites for a mobile client.
On the positive side little else would need to be changed in our provided work as Golang can be executed on both of the most popular mobile operating systems in use currently, Android and iOS.
Indeed the entire application could be written in Golang, building on our already completed work, by utilizing the \emph{mobile} package~\cite{web:site:golang:mobile}.

The web interface peer would be an interesting challenge.
It would allow the user to log in to a web server and enter their passphrase.
The web peer would then be capable of fetching and decrypting the model file and allow the user to upload and download encrypted data directly from an encrypted peer.
This could happen entirely without requiring the underlying Tox communication architecture.
All data to and from the web server would be fully encrypted since decrypted data would only be kept on the web interface peer.
The moment the user closes the web page all temporary data would be discarded.
This would enable users to access their data stored in the Tinzenite network anywhere where they have internet access, as long as they have encrypted peers that enable this feature.

Finally a cold storage peer could be offered.
Technically it would not be a true peer of the Tinzenite network, but for the sake of this text we will reference it as a passive peer.
A user could command a trusted Tinzenite peer to utilize a storage location as a passive peer.
Tinzenite would then encrypt all local files and its current state and write the data to the location.
This location could be anything from a USB stick to more permanent storage device.
If the users wish to update the passive peer they would not even have to do it at the same peer: any other trusted peer could be used too.
This other trusted peer would, much like how the encrypted peer works, read and synchronize the passive peer against itself.
This would allow passive peers to be used both as safeguards against data lose and even as secure transport containers.
Two trusted peers not connected via the internet could be kept synchronized manually by regularly moving a passive peer between them.

\subsubsection{File Versioning}
\label{subs:File Versioning}

%TODO: rework wording. Also: requires storing base copy of file plus all following delta differences to ensure that past versions are accessible.

Another advanced feature that would be very nice to have and close the gap of feature parity between Tinzenite and other existing services would be the capability to offer file versions built directly into the protocol.
Indeed the core protocol would not even have to be changed to support this: all that is required is the capability to keep old files for a specific time somewhere where the peer can reinstate them if the user wishes it.

Therefore all that is needed is a definition of space where these old files can be stored.
We propose placing these files in a further directory labeled \textit{"versions"} within the \textit{.tinzenite} directory.
For each file object a directory with the name of the file identification is created as required where old versions can be stored.
To differentiate the multiple copies each file must be uniquely named: we propose a simple number for each version that increments.

It is important to be able to limit the amount of versions stored to keep the size of the directory manageable.
We believe that this could simple be done by telling any single client how many different copies of files are to be stored, for example the last six.
If a new version is added and it exceeds this self imposed limit the peer can simply trigger a normal removal, thus deleting the version from all peers once the removal has propagated.

Further work could then be done to decrease the size of the versions by only keeping one base copy and storing only binary deltas.
Again this would require somewhat extensive work on peer logic but no change in the core protocol.
To differentiate full copies and delta files we propose an identifying name change would suffice: by appending a special marker behind the version number we could differentiate the type.
The actual synchronization of deltas or full binaries is again already fully implemented in the core protocol and would work as any other file.

%TODO: all stuff below here is WIP and should be rewritten / integrated
\section{TODO}
\label{sec:TODO}

% This is another feature... but where to put it? Expanding I think because it requires both rework and additional work?
The better solution would be to introduce a third value for the shadow attribute: temp.
This value is to be set when the peer knows of an update and has applied it to its model but has not fetched the binary file update yet.
Note that this is a local value only and should not be sent to other peers: in that case all instances of temp should be removed with true.
When connecting to other peers the peer can then request all available files marked as temp and fully update them as required.
This approach has the added advantage of allowing a peer to notify the user that the file has been updated but not fetched yet.
We may also use this functionality to signal slow network transfer speeds: the user could at a glance see that the update is pending, possibly avoiding unnecessary conflicts.

\subsection{Ignored Objects}
\label{sub:Ignored Objects}

%TODO where to put this?
To allow a user to only synchronize objects that they consider important another advanced feature we would like to implement is the ability to specify files and directories to be ignored.
An example for how we propose to implement this feature is the Git source control management software~\cite{web:site:git}.
By specifying the paths to directories and files within a so called \textit{.gitignore} visibility of these can be finely controlled.

In Tinzenite we plan to implement this functionality in a similar manner.
The file containing these rules can be synchronized just as any other file at any level within a directory.
Any new objects detected by Tinzenite that are listed in this file will not be created within the model, effectively keeping it out of the system.
Since it may well happen that such a file is created at a later point and thus introduce an uncertainty in handling the to be ignored files we propose a simple solution: the file is only ever applied for object creations.
Once a file has been created within the model it will be modified and deleted as any other file within the system, independent on the ignore file.

This in turn however means that renaming or moving a file in Tinzenite will remove it from the model, deleting it everywhere.
Since we believe this to be a feature for more advanced users we will not make this feature a core aspect of the system.
Normal users should have little need for only synchronizing a selected subset of files.
