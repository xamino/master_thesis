\chapter{Results and Recapitulation}
\label{chap:Results and Recapitulation}

This chapter will discuss the proof of concept implementation and recapitulate on how the architecture was implemented.
We will then discuss improvements and future work.

\section{Comparison}

In this section we will briefly and informally compare Tinzenite with existing solutions.
Note that an actual study comparing our solution with other storage providers was not the intended goal of this thesis.
All information used to compare the different solutions is to be taken with a grain of salt.

\subsection{Performance}
\label{sub:Performance}

Tinzenite shares a large advantage with the other related peer to peer solutions that have been discussed.
Because the peers communicate directly with each other the synchronization speed for peers within the same local area network is substantially higher than the server-client solutions that must first upload data to a remote server.
In the local area network where Tinzenite was mainly developed the only speed limit was that of the wireless internet for data transfers.

When transferring data between two peers over the internet at large Tinzenite is restricted by the broadband upload speed of the user's internet connection.
While generally speaking the upload speed is only a fraction of the download speed for most consumer connections, Tinzenite does not suffer under this.
This due to that when comparing transferring a file via Tinzenite and any server hosted service, both require an initial upload of the file.
Tinzenite actually wins in terms of speed because by the time the upload has been completed, the other Tinzenite peer already has the complete file downloaded.
Server-client alternatives must still download the file to the other client additionally.

%TODO: add some sort of example, possibly in graph form?
%Nah, just add a text example.
%For that move a large file between two devices with Tinzenite and with GDrive.
%
%TODO: do an informal comparison to previously highlighted comparable software.
%Should include hard facts such as speed (internet vs intranet) and limitations.

\subsection{Usability}
\label{sub:Usability}

Here Tinzenite in its current state definitely loses out to existing solutions.
While running the software requires little to no user interaction, the setup and management of peers is currently only available via command line interface programs.
Nothing speaks against implementing a graphical user interface in principle, this was not an essential part of our work and no such client was implemented due to time constraints.
And while Tinzenite theoretically supports web access to encrypted peers (see section~\ref{subs:Additional Peer Versions} below), this capability was also not tested or implemented.
We believe that web access would improve usability by a large margin and allow Tinzenite to close the feature gap.

Tinzenite's features are complete in that it can and does synchronize directories correctly.
However further work should be done for edge cases where the usability could be improved, such as for merge conflicts.
By notifying and offering support in resolving merge conflicts Tinzenite could provide a more active support for its usability.
Such features could easily be integrated into a client program without requiring modifications to the underlying Tinzenite code.

A feature important for the security of a Tinzenite network is support of clients for strong passphrases.
Users should be guided to create long and secure passphrases because then they become increasingly harder to guess.

\subsection{Security}
\label{sub:Security}

Tinzenite's security is built on one of the most scrutinized encryption libraries currently in existence, the NaCl library.
Therefore the encrypted data that untrusted, encrypted peers receive should be more than sufficient to deny unauthorized third party access.
Should services running encrypted peers be requested by hostile governments to hand over user data said data would be inaccessible without guessing the passphrase used to encrypt the encryption keys.
Thus the security of the user data primarily depends on the user's willingness to use a sufficiently complex passphrase.

Theoretically this feature is Tinzenite's distinguishing point between all existing implementation mentioned in section~\ref{sec:Existing Software}.
Like other peer to peer solutions Tinzenite works directly between peers with most of the associated advantages.
Additionally Tinzenite allows for off site backup of the users' data.
Unlike the server-client solutions Tinzenite does not require an account or trust in a third party.
We believe that Tinzenite therefore combines the best of both worlds in regards to security while retaining readily available access to user stored data for authorized entities.

%disclaimer :P
None the less, Tinzenite should not be trusted with secure data at this point in development.
Primarily because it was written as a proof of concept and thus is probably not bug free.
Since encryption is hard to implement correctly we are also not confident enough of our implementation of the various aspects of the encryption scheme.
Furthermore Tinzenite should only be trusted after a security audit has been performed on the core logic.
Tinzenite's open source nature however allows any interested party to audit the code at their own leisure.
Tinzenite does leak some information: namely the peer list is available unencrypted along with the authentication file.
While a small risk as no true user information is stored in those files it is still a possible attack vector for a malicious peer.

\section{Future Work}
\label{sec:Future Work}

Tinzenite offers a lot of room for future work, both concerning the current implementation and expanding on the provided work.
This section will serve to discuss many of the points we believe could greatly benefit any further work on Tinzenite.
Thus we will begin this section by discussing changes to the current Tinzenite architecture and implementation.
We will conclude this section by offering an outlook of further work that could be based on the existing implementation.

\subsection{Improvements to Existing Work}
\label{sub:Improvements to Existing Work}

Since our implementation of the architecture was built from the ground up without previous knowledge of many of the aspects touched on by this work, we encountered a number of things that should and could be improved.

\subsubsection{Data Transfer}
\label{subs:Data Transfer}

We believe our implementation not to be fully optimized for the best possible data transfer characteristics between a peer network.
Thus improvements to how Tinzenite fetches and sends data could be implemented to improve robustness and speed of data transfers.

In our implementation when a trusted peer receives an update, the associated file is fetched from the peer where the update originated from.
If the update is received from multiple peers only the first peer is used to fetch the data.
Building on the same ideas that led to the development of the BitTorrent protocol, we could implement swarm fetching capabilities.
This would allow the unchoking of saturated peers where upload speeds are slow and allow the file request to complete even if one peer out of many goes offline or encounters issues.

Issues may arise for implementing this for two reasons.
If delta fetching is implemented as described below, care has to be taken to keep the swarm behavior compatible even if multiple peers have varying states of the original file.
Another source of possible issues is how to combine it with the existing encrypted peer behavior.
Since encrypted peers are currently locked to a single trusted peer for a synchronization this precludes having them partake in a swarm apart from the issue that the encrypted peer will transfer encrypted data while trusted peers will transfer unencrypted data.

\subsubsection{Peer Behavior}
\label{subs:Peer Behavior}

Trusted peers of the current Tinzenite implementation synchronize with timing based intervals.
While this works sufficiently for the proof of concept, ideally it would not be the case.
Instead peers should dynamically adjust the timing and order of operations for when to synchronize based on the network environment and current status of the own directory.

An example for this includes pausing the remote synchronization interval if outside changes are still being fetched to avoid unnecessary double fetch requests and associated work.
Another example would be extending the Tinzenite \emph{core} package to allow for finer control over which peers will synchronize and then implementing the client program so that peer synchronizations happen in a smarter fashion.
To reduce the work load for programs building on Tinzenite this could even be implemented within Tinzenite itself.
This could include synchronizing only once when initially connected and then simply updating locally, avoiding unnecessary complete model comparisons.
Synchronization with encrypted peers could also be improved to avoid locking multiple encrypted peers at once and ensure that they are kept up to date at a reasonable rate.

\subsubsection{Delta Updates}
\label{subs:Delta Updates}

Fetching a file in Tinzenite currently always transfers the complete binary data, even if only a small part of the file was changed between two versions.
Thus an improvement would be to implement that Tinzenite only sends the differences between two versions of a file between peers.
We propose to use rsync algorithm for this~\cite{tridgell1996rsync}, specifically the librsync implementation~\cite{web:site:librsync}.
The required information for the library to work can be integrated into the existing request messages.
Delta updates could only be used between trusted peers since the encrypted data is completely different for every change.
Only needing to send the changed part of a file should increase the speed of Tinzenite transfers immensely, expecially for large files.

\subsubsection{Server Peer}
\label{subs:Server Peer}

The current implementation for the encrypted peer was implemented for a single Tinzenite network instance.
It could be extended to provide service for multiple Tinzenite networks and multiple users.
User accounts can be differentiated by reading parts of the authentication file: the user name can be checked with the bcrypt hash.
Care should be taken to ensure that the user does not provide the same access password like the passphrase used to access the Tinzenite encryption, although this won't be enforceable by Tinzenite.

Another feature that the server client should be capable of is enforcing potential size restrictions.
This feature may also be used for a future mobile client as described in section~\ref{subs:Additional Peer Versions}.
What this means is that Tinzenite should support clients refusing to fetch additional files to enforce a specified size of a directory.
It could be implemented by building on the shadow files capability which we will expand on in the next section.
The interesting case is of course what happens to files that are above the limit after a modification: we propose either making the file a shadow file as soon as it crosses the limit or allowing modifications to push the size above the limit temporarily.
For encrypted peers the enforcement of size restrictions must be handled by the trusted peers.
This in turn means that encrypted peers must be capable of denying additional updates since trusted peers can not be trusted to correctly enforce a size limitation.

\subsubsection{Shadow Objects}
\label{subs:Shadow Objects}

Depending on the location of a client a user may wish to only access specific files without having to get an entire set or updates.
This is a nice feature to have in the case of space and bandwidth restricted devices such as mobile devices or for the web interface.
This feature could also be used to prioritize which objects Tinzenite will fetch first.

Functionality for the shadow file feature is available via the currently unused \textit{"shadow"} attribute.
It affects only files directly as the creation of directories is not significant from a size point of view.
The attribute only serves as a shortcut to set all files of a directory implicitly to being shadowed.
If files are marked as shadow files they are not updated on the disk, only their model.
By setting the shadow flag to true the client will then immediately try to fetch the binary file from connected and available peers.

Shadow files pose a few additional difficulties that would have to be solved.
First and most trivial: what happens to an already synchronized file when the shadow attribute is set?
We propose that the file is immediately removed although this could be expensive in terms of bandwidth if the users quickly change their mind again because the file must then be fetched again all anew.
A more sophisticated approach would integrate the size restriction capability of the client as proposed previously.
By setting the space limit to a number below the full size of the directory, files will only be immediately removed if near the space limit.
If the users change their mind the file may thus still be immediately available.

So what do peers do if they receive a model update where the shadow flag is set?
It is important to note here that the shadow flag is considered to be transient when synchronizing models, meaning its value is considered to be local only.
However it is still sent because it is used to determine whether the receiving peer can fetch an update from the other peer if applicable.
Again it is up to the peer what happens upon receiving a shadow file update: normally a peer that has a non shadow copy of the file will ignore shadow updates as it can not fetch the binary file update successfully from it.
It will then have to wait for another peer to offer the update where the attribute is not set.

The final edge case is a challenging one: what Tinzenite does not provide is a way to ensure that one full copy of the shadowed file is always kept somewhere.
If the user marks a file as shadowed on all peers it may well happen that Tinzenite loses the file.
For now we propose to avoid this by explicitly warning the user of this possibility.
One way to mitigate this risk is by allowing user defined shadowed files only for specific clients: we can probably assume that any full desktop peer should always retain a full copy of the directory anyway.
Thus shadow files should only be used for the proposed mobile peer and web interface peer.

\subsubsection{Implementation Improvements}
\label{subs:Implementation Improvements}

Our current implementation, while working, is surely not the best way to implement it.
Not all possible error cases are handled in the most optimal way possible.
Furthermore the implementation of Tinzenite could use extensive testing and debugging.

\subsection{Expanding Work}
\label{sub:Expanding Work}

Tinzenite offers a lot of room to build on.
In this section we will discuss some proposals for building on our work instead of modifying it.
Note that some expanded features require features from the previous section.

\subsubsection{Encrypted Peer Synchronization}
\label{subs:Encrypted Peer Synchronization}

Each encrypted peer is a single point of back up in the current state of our work.
This requires each trusted peer to check up on each encrypted peer in turn and ensure that all of them are kept at the same synchronization state.
This could be improved by allowing encrypted peers to update each other.
Care has to be taken that encrypted peers will not cause merge conflicts that they can not resolve, since both the model and its objects are fully encrypted.
We thus propose to extend the Tinzenite architecture for all encrypted peers to work as a single meta peer, where in truth multiple instances are kept in the same, most up to date state possible.

This would require some further logic in how to detect and merge different encrypted peer states.
Generally it could be done if the directory model had an unencrypted version state attached to it.
As long as no model conflicts arise, all encrypted peers can keep updating themselves to the most current version.
If however two encrypted peers receive conflicting update states, the entire swarm of encrypted peers would need to wait for a trusted peer to resolve the issue.
This would require some large extensions to the current protocol, but should in theory be doable.

\subsubsection{Data Obfuscation}
\label{subs:Data Obfuscation}

While encrypted peers only store encrypted files, simply encrypting a file may not be sufficient to prevent meta information collection on the directory contents.
Thus we propose that future work could include modifying the encrypted peer implementation and the transfer protocol so that trusted peers send not only encrypted but also obfuscated data to encrypted peers, effectively implementing an oblivious storage system~\cite{goldreich1996software}.

This especially makes sense if combined with the swarm behavior mentioned previously and the encrypted peer synchronization.
The entire group of encrypted peers could then be used to obfuscate and store data redundantly.
This would increase the third party security and further reduce the trust of said party required.
Obfuscation would be sufficient to hide most meta data that may be deduced from encrypted synchronization.

\subsubsection{Update Feedback}
\label{subs:Update Feedback}

While our implementation allows for a basic feedback of large file transfers in the form of basic progress meters, a lot more detailed and better specified feedback could be offered to peers.
This would allow them to signal to the user for example of outstanding updates -- such as updates that it has received but not yet applied because the transfer is pending.
Generally such features would also tie in with the better control of peer behavior previously mentioned.
If the user can quickly see at a glance the general state of the local trusted peer or even other connected peers, usability of the entire Tinzenite network increases.
It would also hasten the user's trust in the fully working synchronization.

\subsubsection{Additional Peer Versions}
\label{subs:Additional Peer Versions}

Our current implementation of Tinzenite offers two peers: a standard trusted peer for a personal computer and a peer for encrypted storage.
As discussed in section~\ref{sec:Software Scope} we have already considered adding a mobile client for smartphones and a web interface client for temporary access to an encrypted peer, plus a passive peer for secure cold storage.

A mobile peer would be a trusted active peer of the Tinzenite network.
As previously discussed however the mobile peer would most likely not retain a full copy of the Tinzenite directory due to size and bandwidth limitations.
Thus both shadow objects and size restrictions are likely prerequisites for a mobile client.
On the positive side little else would need to be changed in our provided work as Golang can be executed on both of the most popular mobile operating systems in use currently, Android and iOS.
Indeed the entire application could be written in Golang, building on our already completed work, by utilizing the \emph{mobile} package~\cite{web:site:golang:mobile}.

The web interface peer would be an interesting challenge.
It would allow the users to log in to a web server and enter their passphrase.
The web peer would then be capable of fetching and decrypting the model file and allow the user to upload and download encrypted data directly from an encrypted peer.
This could happen entirely without requiring the underlying Tox communication architecture.
All data to and from the web server would be fully encrypted since decrypted data would only be kept on the web interface peer.
The moment a user closes the web page all temporary data would be discarded.
This would enable users to access their data stored in the Tinzenite network anywhere where they have internet access, as long as they have encrypted peers that enable this feature.

Finally a cold storage peer could be offered.
Technically it would not be a true peer of the Tinzenite network, but for the sake of this text we will reference it as a passive peer.
A user could command a trusted Tinzenite peer to utilize a storage location as a passive peer.
Tinzenite would then encrypt all local files and its current state and write the data to the specified location.
This location could be anything from a USB stick to more permanent storage device.
If the users wish to update the passive peer they would not even have to do it at the same peer: any other trusted peer could be used too.
This other trusted peer would, similar to how the encrypted peer works, read and synchronize the passive peer against itself.
This would allow passive peers to be used both as safeguards against data loss and even as secure transport containers.
Two trusted peers not connected via the internet could be kept synchronized manually by regularly moving a passive peer between them.

\subsubsection{File Versioning}
\label{subs:File Versioning}

Another advanced feature that would be very nice to have and close the gap of feature parity between Tinzenite and other existing services would be the capability to offer file versions built directly into the Tinzenite architecture.
Indeed the core protocol would not even have to be changed to support this: all that is required is the capability to keep old files for a specific time somewhere where the peer can reinstate them if the user wishes to.

We propose to implement this as follows.
First, for every created object that Tinzenite detects, it copies the initial version into a hidden directory for future reference.
Then, whenever a change within the file is detected, the delta update functionality can be used to store the difference between the old version and the new version of the file.
A file removal could be marked specifically.
A setting could be introduced to allow the user to control how far back different versions of an object are kept to keep storage requirements at a reasonable level.
Thus it should be comparatively easy to only retain the last three versions of an object.
Trusted peers could then offer assistance in actually managing the versions if the user wants to reinstate one over the current object.
